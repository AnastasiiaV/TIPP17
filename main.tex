\pdfoutput=1 % only if pdf/png/jpg images are used
\documentclass[a4paper,10pt]{article}
%\documentclass[a4paper,10pt]{scrartcl}
%\documentclass[smallcondensed,referee]{svjour3}
\usepackage{geometry}
\geometry{a4paper,left=34mm,right=34mm, top=26mm, bottom=32mm}
%\geometry{a4paper}
\raggedbottom
\widowpenalty=10000
\clubpenalty=10000 
\usepackage{graphics}  
\usepackage{graphicx}
\usepackage{subfigure}
\usepackage{amsmath}
\usepackage[amssymb]{SIunits}
%\usepackage[load-configurations=abbreviations,tight-spacing=true,separate-uncertainty,bracket-numbers = false]{siunitx} % JINST cannnot handle siunitx !!
\usepackage{nicefrac}
\usepackage[english]{babel}
\usepackage{lineno}
\usepackage{epstopdf}
\usepackage{stfloats}
%\usepackage{upgreek}
\usepackage{verbatim}
\usepackage{url}
\usepackage{xcolor}
\usepackage{url}
%\usepackage{draftwatermark}
%\SetWatermarkScale{5}
\usepackage{setspace}


%general stuff 
\newcommand{\e}{\ensuremath{\mathnormal{e}}}
\newcommand{\h}{\ensuremath{\mathnormal{h}}}
\newcommand{\eV}{\ensuremath{\mathnormal{eV}}}
\newcommand{\cspeed}{\ensuremath{\mathnormal{c}}}

%tscope specific
\newcommand{\DESY}{\ensuremath{\textrm{DESY}}}
\newcommand{\Datura}{\ensuremath{\textrm{DATURA}}}
\newcommand{\Duranta}{\ensuremath{\textrm{DURANTA}}}
\newcommand{\Mimosa}{\ensuremath{\textrm{MIMOSA\,26}}}
\newcommand{\noise}{\ensuremath{\xi_{\textrm{n}}}}
\newcommand{\epsdut}{\ensuremath{\mathnormal{\varepsilon_{\textrm{DUT}}}}}
\newcommand{\epsmimo}{\ensuremath{\mathnormal{\varepsilon_{\textrm{M26}}}}}
\newcommand{\dz}{\ensuremath{\textrm{d}z}}
\newcommand{\dzdut}{\ensuremath{\textrm{d}z_{\textrm{DUT}}}}

%resolutions
\newcommand{\sigmap}{\ensuremath{\sigma_{\textrm{pointing}}}}
\newcommand{\sigmatu}{\ensuremath{\sigma_{\textrm{t,u}}}}
\newcommand{\sigmatb}{\ensuremath{\sigma_{\textrm{t,b}}}}
\newcommand{\sigmat}{\ensuremath{\sigma_{\textrm{t}}}}
\newcommand{\sigmapGBL}{\ensuremath{\sigma_{\textrm{p,GBL}}}}
\newcommand{\sigmameas}{\ensuremath{\sigma_{\textrm{meas}}}}
\newcommand{\sigmadut}{\ensuremath{\sigma_{\textrm{DUT}}}}
\newcommand{\sigmai}{\ensuremath{\sigma_{\textrm{int}}}}
\newcommand{\sigmam}{\ensuremath{\sigma_{\textrm{M26}}}}
\newcommand{\sigmahat}{\ensuremath{\hat{\sigma}_{\textrm{int}}}}

%positions
\newcommand{\zdut}{\ensuremath{z_{\textrm{DUT}}}}
\newcommand{\zzz}{\ensuremath{z_{3}}}

%residuals
\newcommand{\rbiased}{\ensuremath{r_{\textrm{b}}}}
\newcommand{\runbiased}{\ensuremath{r_{\textrm{u}}}}
\newcommand{\rhat}{\ensuremath{\hat{r}_{\textrm{b}}}}
\newcommand{\pb}{\ensuremath{p_{\textrm{b}}}}


%software
\newcommand{\eudet}{\ensuremath{\textrm{EUDET}}}
\newcommand{\rawdataevent}{RawDataEvent }
\newcommand{\rawdataevents}{RawDataEvents }
\newcommand{\eudaq}{\ensuremath{\textrm{EUDAQ}}}
\newcommand{\EUTelescope}{\ensuremath{\textrm{EUTelescope}}}

%conditional compilation
\newcommand*{\notFOREPJ}{}%

%linenumbers
\setpagewiselinenumbers
\modulolinenumbers[5]

\ifdefined\notFOREPJ
\else
\doublespacing
\fi

\makeatletter
\renewcommand{\maketitle}{\bgroup\setlength{\parindent}{0pt}
\begin{flushleft}
  \vspace*{10mm}
  \textbf{\huge\sffamily\@title}
  \vspace{5mm}
   
  \large \@author
\end{flushleft}\egroup
}
\def\@xfootnote[#1]{%
  \protected@xdef\@thefnmark{#1}%
  \@footnotemark\@footnotetext}
\makeatother

%\setlength\extrarowheight{5pt}
\renewcommand{\arraystretch}{1.25}


%\keywords{implants, R&D, resolution, tracker} % need to be specified during submission

\begin{document}
\ifdefined\notFOREPJ
%\linenumbers
\else
\fi



%DESY 6
%D. Eckstein, 				doris.eckstein@desy.de
%T. Eichhorn, 				thomas.eichhorn@desy.de
%I. Gregor,				ingrid.gregor@desy.de
%C. Muhl,				carsten.muhl@desy.de
%H. Jansen, 				hendrik.jansen@desy.de
%R. Peschke,				Richard.Peschke@desy.de
%S. Spannagel,				simon.spannagel@desy.de

% Uni of Bristol 1
%D. G. Cussansm				David.Cussans@bristol.ac.uk

%DPNC 2
%E. Corrin,		SwiftKey	emlyn.corrin@gmail.com
%D. Hass, 		SRON: 		D.Haas@sron.nl

%IPHC Strasbourg, France 3
%Mark Winter				marc.winter@iphc.cnrs.fr
%Mathieu Goffe				mathieu.goffe@iphc.cnrs.fr
%Gilles Claus				gilles.claus@iphc.cnrs.fr 

%INFN Como 1
%Antonio Bulgheroni, 	KIT		antonio.bulgheroni@gmail.com

%Ex-DESY: 5
%H. Perrey, 				hanno@perrey.info
%Philipp Roloff,	CERN 		philipp.roloff@cern.ch
%I. Rubinskiy,		CFEl/CUIigor	igor.rubinskiy@cfel.de

\title{Enhanced lateral drift sensors: concept and development}
\author{
A.~Velyka${}^{\textrm{a,}}$\footnote[*]{Corresponding author: anastasiia.velyka@desy.de}, %
H.~Jansen${}^{\textrm{a,}}$\footnote[**]{Corresponding author: hendrik.jansen@desy.de}%,
%S.~Spannagel${}^{\textrm{a}}$, 
%J.~Behr${}^{\textrm{a,}}$\footnote{Now at Institut f\"ur Unfallanalysen, Hamburg, Germany},
%A.~Bulgheroni${}^{\textrm{b,}}$\footnote{Now at KIT, Karlsruhe, Germany},
%G.~Claus${}^{\textrm{c}}$,
%E.~Corrin${}^{\textrm{d,}}$\footnote{Now at SwiftKey, London, UK},
%D.~G.~Cussans${}^{\textrm{e}}$,
%J.~Dreyling-Eschweiler${}^{\textrm{a}}$, 
%D.~Eckstein${}^{\textrm{a}}$, 
%T.~Eichhorn${}^{\textrm{a}}$, 
%M.~Goffe${}^{\textrm{c}}$,
%I.~M.~Gregor${}^{\textrm{a}}$, 
%D.~Haas${}^{\textrm{d,}}$\footnote{Now at SRON, Utrecht, Netherlands},
%C.~Muhl${}^{\textrm{a}}$,
%H.~Perrey${}^{\textrm{a,}}$\footnote{Now at Lund University, Sweden}, 
%R.~Peschke${}^{\textrm{a}}$, 
%P.~Roloff${}^{\textrm{a,}}$\footnote{Now at CERN, Geneva, Switzerland}, 
%I.~Rubinskiy${}^{\textrm{a,}}$\footnote{Now at CFEL, Hamburg, Germany}, 
%M.~Winter${}^{\textrm{c}}$
\\
\vspace{3mm}
${}^{\textrm{a}}$ Deutsches Elektronen-Synchrotron DESY, Hamburg, Germany\\
%${}^{\textrm{b}}$ INFN Como, Italy\\
%${}^{\textrm{c}}$ IPHC, Strasbourg, France\\
%${}^{\textrm{c}}$ D\'epartement de physique nucl\'eaire et corpusculaire, University of Geneva, Switzerland\\
%${}^{\textrm{d}}$ DPNC, University of Geneva, Switzerland\\
%${}^{\textrm{e}}$ University of Bristol, UK
\vspace{3mm}
}
\maketitle




\begin{abstract}
\noindent
%Detailed studies of the resolution of a $\eudet$-type beam telescope are carried out using the $\Datura$ beam telescope as an example. 
%The $\eudet$-type beam telescopes make use of CMOS $\Mimosa$ pixel detectors for particle tracking allowing for precise characterisation of particle-sensing devices. 
%A profound understanding of the performance of the beam telescope as a whole is obtained by a detailed characterisation of the sensors themselves. 
%The differential intrinsic resolution as measured in a $\Mimosa$ sensor is extracted using an iterative pull method, and various quantities
% that depend on the size of the cluster produced by a traversing charged particle are discussed:
% the residual distribution, the intra-pixel residual-width distribution and the intra-pixel density distribution of track incident positions.\\

Development of the ELAD sensors for increasing the charge sharing. The ELAD sensors make use of deep implants for creating the non-homogeneous electric field. In order to find an optimal sensor design, detailed simulation studies have been conducted using SYNOPSYS TCAD. For specifying sensors design 2 types of simulations were done: simulation of the electric field and drift simulations. A description of the multi-layer production process is presented, which represents a new production technique allowing for deep bulk engineering.  



\noindent
Keywords: implants, R$\&$D, resolution, tracker, TCAD % need to be specified during submission

%Additionally, we discuss intra-pixel residuals the differential intrinsic resolution and the angle resolution of the beam telescope used. 
%Charge collection simulations of the sensor are compared to the measured results. 
%Finally, the amount of angular scattering in aluminium targets for thicknesses from as small as $\SI{10}{\um}$ to $\SI{500}{\um}$ is measured and compared to predictions and simulations. 
\end{abstract}

% Abstract with no \commands unless numbers for submission process


% \ifdefined\notFOREPJ
% \tableofcontents
% \else
% \fi


\section{Introduction}
\label{sec:intro}
\ifdefined\notFOREPJ
One of the main goals in the R$\&$D of tracker sensors technology is to increase the position resolution of the particle detector. There are two ways to achieve this. The most common way is to decrease the size of the read-out cell, i.e. to decrease the pixel or strip pitch. But in this case, the number of channels increases, which requires an increased bandwidth for the read-out. The other possibility to improve the position resolution of sensors is to increase the lateral size of the charge distribution already during the drift in the sensor material. In this case, it is necessary to carefully engineer the electric field in the bulk of this so-called enhanced lateral drift (ELAD) sensor. This new design is uses implants deep inside of the bulk. Implants constitute volumes with different values of doping concentration in comparison to the concentration in the bulk. This allows for modification of the drift path of the charge carriers in the sensor.
\else
 One of the main goals in the R$\&$D of tracker sensors technology is to increase the position resolution of the particle detector. There are two ways to achieve this. The most common way is to decrease the size of the read-out cell, i.e. to decrease the pixel or strip pitch. But in this case, the number of channels increases, which requires an increased bandwidth for the read-out. The other possibility to improve the position resolution of sensors is to increase the lateral size of the charge distribution already during the drift in the sensor material. In this case, it is necessary to carefully engineer the electric field in the bulk of this so-called enhanced lateral drift (ELAD) sensor. This new design is uses implants deep inside of the bulk. Implants constitute volumes with different values of doping concentration in comparison to the concentration in the bulk. This allows for modification of the drift path of the charge carriers in the sensor.
\fi
% 
% \section{Beamlines}
% \label{sec:beamlines}
% \ifdefined\notFOREPJ
%  \input{content/beam_lines}
% \else
%  \input{beam_lines}
% \fi

\section{Concept}
\label{sec:concept}
\ifdefined\notFOREPJ
% 
In the semiconductor sensors a charged particle creates electron-hole pairs which drift under the influence of an electric field to the read-out electrodes. If all charge carriers are collected by a single strip/pixel, the position resolution depends only on the distance between strips/pixels, i.e. on pitch size. If the lateral broadening of the charge is compared with the strip/pixel pitch, the part of the charge may be collected at a neighboring strip/pixel. This effect, called charge sharing, can improve the accuracy of the position resolution. The charge sharing could be achieved by using the magnetic field, by tilting the sensor or increasing the lateral size of the charge distribution during the drift.

The charge carriers drift along the lines of electric field. In order to spread the charge in the lateral direction, the electric field should be modified. To manipulate the electric field, deep implants are used. The p+ implants in a p-bulk sensor are conceived by incorporation of Boron atoms, and  create a repulsive area for electrons. When the electron cloud meets a p+ implant on the drifting way, the former splits into two halves fig.1 (left). Applying it layer-wise gives the charge sharing between three strips/pixels fig.1 (right).  The charge cloud is split 50/50 on every layer, which gives a binomial distribution of the charge.

\begin{figure}[h]
\center{\includegraphics[width=12cm]{/Users/tweety/Documents/PhD/TIPP/TIPP17/pictures/binomial.png}}

Fig. 1. The example of the 50/50 charge sharing in the ELAD sensor. 
\end{figure}

The binomial charge distribution gives a cluster size of 3 that is not optimal. The neighboring strips may collect not enough charge to pass the sensor's threshold. More efficient charge sharing is achieved at a value of cluster size of 2. In order to obtain this value of the cluster size, it is necessary to change the position of the implants. Usage of p+ deep implants leads to the increase of the value of the doping concentration in the sensor. This causes to increasing of the value of $N_{eff}$. The high value of $N_{eff}$ induces the increase of the depletion voltage. 

\begin{equation}
N_{eff}=N_{D}-N_{A}
\end{equation}

\begin{equation}
V_{depl}=\frac{q_0 D^2 |N_{eff}|}{2 \epsilon_r \epsilon_0 }
\end{equation}

To control the value of $N_{eff}$ deep p+ and n+ implants are used. The implants are located between 2 strips/pixels, which gives a cluster size of 2. The n+ implants are placed between two p+ implants. If the doping concentrations of p+ and n+ implants are carefully balanced, the value of $N_{eff}$ and, correspondingly, $V_{depl}$ remains not to high. The number of implant layers should be as small as possible due to the difficult production process. 

\else
 
In the semiconductor sensors a charged particle creates electron-hole pairs which drift under the influence of an electric field to the read-out electrodes. If all charge carriers are collected by a single strip/pixel, the position resolution depends only on the distance between strips/pixels, i.e. on pitch size. If the lateral broadening of the charge is compared with the strip/pixel pitch, the part of the charge may be collected at a neighboring strip/pixel. This effect, called charge sharing, can improve the accuracy of the position resolution. The charge sharing could be achieved by using the magnetic field, by tilting the sensor or increasing the lateral size of the charge distribution during the drift.

The charge carriers drift along the lines of electric field. In order to spread the charge in the lateral direction, the electric field should be modified. To manipulate the electric field, deep implants are used. The p+ implants in a p-bulk sensor are conceived by incorporation of Boron atoms, and  create a repulsive area for electrons. When the electron cloud meets a p+ implant on the drifting way, the former splits into two halves fig.1 (left). Applying it layer-wise gives the charge sharing between three strips/pixels fig.1 (right).  The charge cloud is split 50/50 on every layer, which gives a binomial distribution of the charge.

\begin{figure}[h]
\center{\includegraphics[width=12cm]{/Users/tweety/Documents/PhD/TIPP/TIPP17/pictures/binomial.png}}

Fig. 1. The example of the 50/50 charge sharing in the ELAD sensor. 
\end{figure}

The binomial charge distribution gives a cluster size of 3 that is not optimal. The neighboring strips may collect not enough charge to pass the sensor's threshold. More efficient charge sharing is achieved at a value of cluster size of 2. In order to obtain this value of the cluster size, it is necessary to change the position of the implants. Usage of p+ deep implants leads to the increase of the value of the doping concentration in the sensor. This causes to increasing of the value of $N_{eff}$. The high value of $N_{eff}$ induces the increase of the depletion voltage. 

\begin{equation}
N_{eff}=N_{D}-N_{A}
\end{equation}

\begin{equation}
V_{depl}=\frac{q_0 D^2 |N_{eff}|}{2 \epsilon_r \epsilon_0 }
\end{equation}

To control the value of $N_{eff}$ deep p+ and n+ implants are used. The implants are located between 2 strips/pixels, which gives a cluster size of 2. The n+ implants are placed between two p+ implants. If the doping concentrations of p+ and n+ implants are carefully balanced, the value of $N_{eff}$ and, correspondingly, $V_{depl}$ remains not to high. The number of implant layers should be as small as possible due to the difficult production process. 

\fi

\section{TCAD simulations}
\label{sec:simulations}
\ifdefined\notFOREPJ
% Careful simulation is essential for a new sensor development. The  Technology Computer-Aided Design (TCAD) SYNOPSYS was selected  as a tool for simulations. It is a flexible platform for computer simulations, widely used for the development and optimization of semiconductor processing technologies and devices [1]. In particular, three tools were mainly exploited: SPROCESS, SDE and SDEVICE. SPROCESS simulates the fabrication steps in silicon process technologies. SDE builds and edits device structures using geometric operations. SDEVICE simulates the electrical, thermal, and optical characteristics of silicon and compound semiconductor devices [2].

The parameters which should be clarified are the width and depth of implants, distance within/to next layer, position/shift to neighboring layer, the number of layers and optimal doping concentrations for deep implants. As a simulation result, electric field profile for best charge sharing should be obtained. 

The TCAD geometry of the ELAD sensor (fig.2) is contain p-type sensor, 3 layers of n and p deep implants, p-spray isolation, readout implants, readout electrodes, $SiO_2$ and backplane. With the p-spray isolation technique a shallow unstructured p implant provides a doping density that, integrated over depth, exceeds the oxide surface charge density, thus preventing the buildup of an electron layer below the oxide [3]. The p-spray concentration was chosen according to the value of the breakdown voltage [4]. The shape of implants is described by using an error function. The total sensor thickness is 150 $\mu m$, the pitch size is 55 $\mu m$ in agreement with TimePix3 readout chip. 

\begin{figure}[h]
\center{\includegraphics[width=5cm]{pictures/geometry.png}}

Fig. 2. TCAD geometry of the ELAD sensor. 1 - p-spray, 2 - deep p-implant, 3 - deep n-implant, 4 - epi-zone. 
\end{figure}

The TCAD simulation of an electric field shows that the p and n deep implants lead to cause changes of potential (fig.3). The mean electric field is similar to an electric field in a standard design sensor, but the local higher dose implants change the field locally, i.e. include an electric field component in $x$ direction. Red zones in fig.3 force electrons to change their drift path to the right, blue zones to the left, consequently create the possibility of collecting the charge by two strips. In addition the simulation shows that the non-homogeneous electric field in $x$ direction in the ELAD sensor is stable in time. 

\begin{figure}[h]
\center{\includegraphics[width=6cm]{pictures/elfield.png}}

Fig. 3. TCAD simulation of the electric field in the ELAD sensor. 
\end{figure}

To understand the behavior of charge carriers in the ELAD sensor the drift simulation was done. It has shown that the charge carriers created near an electrode are collected by it, but the real part of charge i.e. the charge created beneath the deep implants area, changes its drift path and is collected by two strips (fig.4).  Thereby, the drift simulation proves that the charge sharing in ELAD sensors is possible on condition of using the non-homogeneous electric field in $x$ direction. 


\begin{figure}[h]
\center{\includegraphics[width=14.2cm]{pictures/drift2.png}}

Fig. 4. TCAD drift simulation in the ELAD sensor. 1 - $t = 10^{-12} s$, 2 - $t = 10^{-10} s$, 3 - $t = 1.2 \cdot 10^{-9} s$. 
\end{figure}


\else
 Careful simulation is essential for a new sensor development. The  Technology Computer-Aided Design (TCAD) SYNOPSYS was selected  as a tool for simulations. It is a flexible platform for computer simulations, widely used for the development and optimization of semiconductor processing technologies and devices [1]. In particular, three tools were mainly exploited: SPROCESS, SDE and SDEVICE. SPROCESS simulates the fabrication steps in silicon process technologies. SDE builds and edits device structures using geometric operations. SDEVICE simulates the electrical, thermal, and optical characteristics of silicon and compound semiconductor devices [2].

The parameters which should be clarified are the width and depth of implants, distance within/to next layer, position/shift to neighboring layer, the number of layers and optimal doping concentrations for deep implants. As a simulation result, electric field profile for best charge sharing should be obtained. 

The TCAD geometry of the ELAD sensor (fig.2) is contain p-type sensor, 3 layers of n and p deep implants, p-spray isolation, readout implants, readout electrodes, $SiO_2$ and backplane. With the p-spray isolation technique a shallow unstructured p implant provides a doping density that, integrated over depth, exceeds the oxide surface charge density, thus preventing the buildup of an electron layer below the oxide [3]. The p-spray concentration was chosen according to the value of the breakdown voltage [4]. The shape of implants is described by using an error function. The total sensor thickness is 150 $\mu m$, the pitch size is 55 $\mu m$ in agreement with TimePix3 readout chip. 

\begin{figure}[h]
\center{\includegraphics[width=5cm]{pictures/geometry.png}}

Fig. 2. TCAD geometry of the ELAD sensor. 1 - p-spray, 2 - deep p-implant, 3 - deep n-implant, 4 - epi-zone. 
\end{figure}

The TCAD simulation of an electric field shows that the p and n deep implants lead to cause changes of potential (fig.3). The mean electric field is similar to an electric field in a standard design sensor, but the local higher dose implants change the field locally, i.e. include an electric field component in $x$ direction. Red zones in fig.3 force electrons to change their drift path to the right, blue zones to the left, consequently create the possibility of collecting the charge by two strips. In addition the simulation shows that the non-homogeneous electric field in $x$ direction in the ELAD sensor is stable in time. 

\begin{figure}[h]
\center{\includegraphics[width=6cm]{pictures/elfield.png}}

Fig. 3. TCAD simulation of the electric field in the ELAD sensor. 
\end{figure}

To understand the behavior of charge carriers in the ELAD sensor the drift simulation was done. It has shown that the charge carriers created near an electrode are collected by it, but the real part of charge i.e. the charge created beneath the deep implants area, changes its drift path and is collected by two strips (fig.4).  Thereby, the drift simulation proves that the charge sharing in ELAD sensors is possible on condition of using the non-homogeneous electric field in $x$ direction. 


\begin{figure}[h]
\center{\includegraphics[width=14.2cm]{pictures/drift2.png}}

Fig. 4. TCAD drift simulation in the ELAD sensor. 1 - $t = 10^{-12} s$, 2 - $t = 10^{-10} s$, 3 - $t = 1.2 \cdot 10^{-9} s$. 
\end{figure}


\fi


\section{Production}
\label{sec:discussion}
\ifdefined\notFOREPJ
% 
The production of the ELAD sensor requires the developing of a new manufacturing technology. 
This new technology involves surface implantation and epitaxial growth. 
One of the methods of epitaxial growth~\cite{Lutz} is the Chemical Vapour Deposition (CVD) method. 
The CVD is the deposition of gaseous material onto a heated substrate. 
The deposition is achieved via the chemical reaction of the substructure with gaseous compounds. 
The surface implantation is realised by bombarding the substrate with accelerated ions.

The ELAD production uses the two techniques alternately. 
Initially, the first layer of deep implants is created by surface implantation on a p-type wafer. 
Then, on the implanted wafer creates an epitaxial layer, in the surface of this epitaxial zone, puts the second layer of deep implants and the process repeats. %FIXME bad english, rephrase

One possible risk of such a production is the ability of implants to diffuse during the CVD process. 
However, the TCAD SPROCESS simulation shows that the difference in sizes of deep implants before and after 20\,min heating up to $1100^\circ$C is less than $\SI{1}{\um}$. 
Therefore, %FIXME finish sentence

\else
 
The production of the ELAD sensor requires the developing of a new manufacturing technology. 
This new technology involves surface implantation and epitaxial growth. 
One of the methods of epitaxial growth~\cite{Lutz} is the Chemical Vapour Deposition (CVD) method. 
The CVD is the deposition of gaseous material onto a heated substrate. 
The deposition is achieved via the chemical reaction of the substructure with gaseous compounds. 
The surface implantation is realised by bombarding the substrate with accelerated ions.

The ELAD production uses the two techniques alternately. 
Initially, the first layer of deep implants is created by surface implantation on a p-type wafer. 
Then, on the implanted wafer creates an epitaxial layer, in the surface of this epitaxial zone, puts the second layer of deep implants and the process repeats. %FIXME bad english, rephrase

One possible risk of such a production is the ability of implants to diffuse during the CVD process. 
However, the TCAD SPROCESS simulation shows that the difference in sizes of deep implants before and after 20\,min heating up to $1100^\circ$C is less than $\SI{1}{\um}$. 
Therefore, %FIXME finish sentence

\fi


\section{Conclusion}
\label{sec:conclusion}
\ifdefined\notFOREPJ
 %ELAD sensor it's a new sensor concept that gives a possibility to achieve a high position resolution without making a smaller pitch or increasing the number of readout channels. The improvement of resolution in ELAD sensors is accomplished by increasing the lateral size of the charge distribution. For this, in the ELAD sensors deep implants are included. The characteristic length affecting the spatial resolution is the distance between the deep implants rather than the pitch size. The simulations show that the charge sharing in ELAD sensors is possible. New production technique for creating multi-layered structures with implantation was developed. 
\else
 ELAD sensor it's a new sensor concept that gives a possibility to achieve a high position resolution without making a smaller pitch or increasing the number of readout channels. The improvement of resolution in ELAD sensors is accomplished by increasing the lateral size of the charge distribution. For this, in the ELAD sensors deep implants are included. The characteristic length affecting the spatial resolution is the distance between the deep implants rather than the pitch size. The simulations show that the charge sharing in ELAD sensors is possible. New production technique for creating multi-layered structures with implantation was developed. 
\fi

%\ifdefined\notFOREPJ
%\else
\section*{Acknowledgements}

{\small
\section*{Data and materials}
%The datasets supporting the conclusions of this article are available from reference \cite{jansen_data}.
%The software used is available from the github repositories: 1) \url{https://github.com/eutelescope/eutelescope}, 2) \url{https://github.com/simonspa/eutelescope/}, branch \textit{scattering}
% and 3) \url{https://github.com/simonspa/resolution-simulator}.
%For the presented analysis, these specific tags have been used: \cite{jansen_2016_49065}.

% \section*{Competing interests}
% The authors declare that they have no competing interests.


\bibliographystyle{IEEEtran}
\ifdefined\notFOREPJ
\bibliography{bibtex/refs}
\else
\bibliography{refs}
\fi
}

\end{document}


\grid
