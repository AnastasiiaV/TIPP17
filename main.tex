\pdfoutput=1 % only if pdf/png/jpg images are used
\documentclass[a4paper,10pt]{article}
%\documentclass[a4paper,10pt]{scrartcl}
%\documentclass[smallcondensed,referee]{svjour3}
\usepackage{geometry}
\geometry{a4paper,left=34mm,right=34mm, top=26mm, bottom=32mm}
%\geometry{a4paper}
\raggedbottom
\widowpenalty=10000
\clubpenalty=10000 
\usepackage{graphics}  
\usepackage{graphicx}
\usepackage{subfigure}
\usepackage{amsmath}
\usepackage[amssymb]{SIunits}
%\usepackage[load-configurations=abbreviations,tight-spacing=true,separate-uncertainty,bracket-numbers = false]{siunitx} % JINST cannnot handle siunitx !!
\usepackage{nicefrac}
\usepackage[english]{babel}
\usepackage{lineno}
\usepackage{epstopdf}
\usepackage{stfloats}
%\usepackage{upgreek}
\usepackage{verbatim}
\usepackage{url}
\usepackage{xcolor}
\usepackage{url}
%\usepackage{draftwatermark}
%\SetWatermarkScale{5}
\usepackage{setspace}


%general stuff 
\newcommand{\e}{\ensuremath{\mathnormal{e}}}
\newcommand{\h}{\ensuremath{\mathnormal{h}}}
\newcommand{\eV}{\ensuremath{\mathnormal{eV}}}
\newcommand{\cspeed}{\ensuremath{\mathnormal{c}}}

%tscope specific
\newcommand{\DESY}{\ensuremath{\textrm{DESY}}}
\newcommand{\Datura}{\ensuremath{\textrm{DATURA}}}
\newcommand{\Duranta}{\ensuremath{\textrm{DURANTA}}}
\newcommand{\Mimosa}{\ensuremath{\textrm{MIMOSA\,26}}}
\newcommand{\noise}{\ensuremath{\xi_{\textrm{n}}}}
\newcommand{\epsdut}{\ensuremath{\mathnormal{\varepsilon_{\textrm{DUT}}}}}
\newcommand{\epsmimo}{\ensuremath{\mathnormal{\varepsilon_{\textrm{M26}}}}}
\newcommand{\dz}{\ensuremath{\textrm{d}z}}
\newcommand{\dzdut}{\ensuremath{\textrm{d}z_{\textrm{DUT}}}}

%resolutions
\newcommand{\sigmap}{\ensuremath{\sigma_{\textrm{pointing}}}}
\newcommand{\sigmatu}{\ensuremath{\sigma_{\textrm{t,u}}}}
\newcommand{\sigmatb}{\ensuremath{\sigma_{\textrm{t,b}}}}
\newcommand{\sigmat}{\ensuremath{\sigma_{\textrm{t}}}}
\newcommand{\sigmapGBL}{\ensuremath{\sigma_{\textrm{p,GBL}}}}
\newcommand{\sigmameas}{\ensuremath{\sigma_{\textrm{meas}}}}
\newcommand{\sigmadut}{\ensuremath{\sigma_{\textrm{DUT}}}}
\newcommand{\sigmai}{\ensuremath{\sigma_{\textrm{int}}}}
\newcommand{\sigmam}{\ensuremath{\sigma_{\textrm{M26}}}}
\newcommand{\sigmahat}{\ensuremath{\hat{\sigma}_{\textrm{int}}}}

%positions
\newcommand{\zdut}{\ensuremath{z_{\textrm{DUT}}}}
\newcommand{\zzz}{\ensuremath{z_{3}}}

%residuals
\newcommand{\rbiased}{\ensuremath{r_{\textrm{b}}}}
\newcommand{\runbiased}{\ensuremath{r_{\textrm{u}}}}
\newcommand{\rhat}{\ensuremath{\hat{r}_{\textrm{b}}}}
\newcommand{\pb}{\ensuremath{p_{\textrm{b}}}}


%software
\newcommand{\eudet}{\ensuremath{\textrm{EUDET}}}
\newcommand{\rawdataevent}{RawDataEvent }
\newcommand{\rawdataevents}{RawDataEvents }
\newcommand{\eudaq}{\ensuremath{\textrm{EUDAQ}}}
\newcommand{\EUTelescope}{\ensuremath{\textrm{EUTelescope}}}

%conditional compilation
\newcommand*{\notFOREPJ}{}%

%linenumbers
\setpagewiselinenumbers
\modulolinenumbers[5]

\ifdefined\notFOREPJ
\else
\doublespacing
\fi

\makeatletter
\renewcommand{\maketitle}{\bgroup\setlength{\parindent}{0pt}
\begin{flushleft}
  \vspace*{10mm}
  \textbf{\huge\sffamily\@title}
  \vspace{5mm}
   
  \large \@author
\end{flushleft}\egroup
}
\def\@xfootnote[#1]{%
  \protected@xdef\@thefnmark{#1}%
  \@footnotemark\@footnotetext}
\makeatother

%\setlength\extrarowheight{5pt}
\renewcommand{\arraystretch}{1.25}


%\keywords{Pixel, CMOS, Beam telescope, resolution, tracker} % need to be specified during submission

\begin{document}
\ifdefined\notFOREPJ
%\linenumbers
\else
\fi



%DESY 6
%D. Eckstein, 				doris.eckstein@desy.de
%T. Eichhorn, 				thomas.eichhorn@desy.de
%I. Gregor,				ingrid.gregor@desy.de
%C. Muhl,				carsten.muhl@desy.de
%H. Jansen, 				hendrik.jansen@desy.de
%R. Peschke,				Richard.Peschke@desy.de
%S. Spannagel,				simon.spannagel@desy.de

% Uni of Bristol 1
%D. G. Cussansm				David.Cussans@bristol.ac.uk

%DPNC 2
%E. Corrin,		SwiftKey	emlyn.corrin@gmail.com
%D. Hass, 		SRON: 		D.Haas@sron.nl

%IPHC Strasbourg, France 3
%Mark Winter				marc.winter@iphc.cnrs.fr
%Mathieu Goffe				mathieu.goffe@iphc.cnrs.fr
%Gilles Claus				gilles.claus@iphc.cnrs.fr 

%INFN Como 1
%Antonio Bulgheroni, 	KIT		antonio.bulgheroni@gmail.com

%Ex-DESY: 5
%H. Perrey, 				hanno@perrey.info
%Philipp Roloff,	CERN 		philipp.roloff@cern.ch
%I. Rubinskiy,		CFEl/CUIigor	igor.rubinskiy@cfel.de

\title{Resolution studies with the DATURA beam telescope}
\author{
H.~Jansen${}^{\textrm{a,}}$\footnote[*]{Corresponding author: hendrik.jansen@desy.de}%,
%S.~Spannagel${}^{\textrm{a}}$, 
%J.~Behr${}^{\textrm{a,}}$\footnote{Now at Institut f\"ur Unfallanalysen, Hamburg, Germany},
%A.~Bulgheroni${}^{\textrm{b,}}$\footnote{Now at KIT, Karlsruhe, Germany},
%G.~Claus${}^{\textrm{c}}$,
%E.~Corrin${}^{\textrm{d,}}$\footnote{Now at SwiftKey, London, UK},
%D.~G.~Cussans${}^{\textrm{e}}$,
%J.~Dreyling-Eschweiler${}^{\textrm{a}}$, 
%D.~Eckstein${}^{\textrm{a}}$, 
%T.~Eichhorn${}^{\textrm{a}}$, 
%M.~Goffe${}^{\textrm{c}}$,
%I.~M.~Gregor${}^{\textrm{a}}$, 
%D.~Haas${}^{\textrm{d,}}$\footnote{Now at SRON, Utrecht, Netherlands},
%C.~Muhl${}^{\textrm{a}}$,
%H.~Perrey${}^{\textrm{a,}}$\footnote{Now at Lund University, Sweden}, 
%R.~Peschke${}^{\textrm{a}}$, 
%P.~Roloff${}^{\textrm{a,}}$\footnote{Now at CERN, Geneva, Switzerland}, 
%I.~Rubinskiy${}^{\textrm{a,}}$\footnote{Now at CFEL, Hamburg, Germany}, 
%M.~Winter${}^{\textrm{c}}$
\\
\vspace{3mm}
${}^{\textrm{a}}$ Deutsches Elektronen-Synchrotron DESY, Hamburg, Germany\\
%${}^{\textrm{b}}$ INFN Como, Italy\\
%${}^{\textrm{c}}$ IPHC, Strasbourg, France\\
%${}^{\textrm{c}}$ D\'epartement de physique nucl\'eaire et corpusculaire, University of Geneva, Switzerland\\
%${}^{\textrm{d}}$ DPNC, University of Geneva, Switzerland\\
%${}^{\textrm{e}}$ University of Bristol, UK
\vspace{3mm}
}
\maketitle




\begin{abstract}
\noindent
Detailed studies of the resolution of a $\eudet$-type beam telescope are carried out using the $\Datura$ beam telescope as an example. 
The $\eudet$-type beam telescopes make use of CMOS $\Mimosa$ pixel detectors for particle tracking allowing for precise characterisation of particle-sensing devices. 
A profound understanding of the performance of the beam telescope as a whole is obtained by a detailed characterisation of the sensors themselves. 
The differential intrinsic resolution as measured in a $\Mimosa$ sensor is extracted using an iterative pull method, and various quantities
 that depend on the size of the cluster produced by a traversing charged particle are discussed:
 the residual distribution, the intra-pixel residual-width distribution and the intra-pixel density distribution of track incident positions.\\

\noindent
Keywords: CMOS pixel sensor, Beam telescope, Intrinsic resolution, Intra-pixel studies % need to be specified during submission

%Additionally, we discuss intra-pixel residuals the differential intrinsic resolution and the angle resolution of the beam telescope used. 
%Charge collection simulations of the sensor are compared to the measured results. 
%Finally, the amount of angular scattering in aluminium targets for thicknesses from as small as $\SI{10}{\um}$ to $\SI{500}{\um}$ is measured and compared to predictions and simulations. 
\end{abstract}

% Abstract with no \commands unless numbers for submission process


% \ifdefined\notFOREPJ
% \tableofcontents
% \else
% \fi


\section{Introduction}
\label{sec:intro}
\ifdefined\notFOREPJ
% 
Future experiments in particle physics require few-micrometer position resolution in their tracking detectors. 
Nowadays silicon is the material of choice for high-precision detectors, as it provides a large variety of engineering possibilities. 
The most common way to achieve a high position resolution in tracking sensors is to decrease the size of the read-out cell, i.e. to decrease the pixel or strip pitch. 
This, however, increases the number of readout channels and requires larger bandwidth. 
Another possibility to improve the position resolution of the sensors is to increase the lateral size of the charge distribution already during the drift in the sensor material. 

In a new sensor concept, instead of decreasing the pitch size, the electric field is changed by implants deep inside the bulk, which force the charge carriers to change their drift path. 
The implants constitute volumes of different doping concentration with respect to the concentration in the bulk. 
In oder to achieve an optimal charge sharing to carefully engineer the electric field in the bulk of the ELAD sensor ~\cite{JANSEN2016242}.

\else
 
Future experiments in particle physics require few-micrometer position resolution in their tracking detectors. 
Nowadays silicon is the material of choice for high-precision detectors, as it provides a large variety of engineering possibilities. 
The most common way to achieve a high position resolution in tracking sensors is to decrease the size of the read-out cell, i.e. to decrease the pixel or strip pitch. 
This, however, increases the number of readout channels and requires larger bandwidth. 
Another possibility to improve the position resolution of the sensors is to increase the lateral size of the charge distribution already during the drift in the sensor material. 

In a new sensor concept, instead of decreasing the pitch size, the electric field is changed by implants deep inside the bulk, which force the charge carriers to change their drift path. 
The implants constitute volumes of different doping concentration with respect to the concentration in the bulk. 
In oder to achieve an optimal charge sharing to carefully engineer the electric field in the bulk of the ELAD sensor ~\cite{JANSEN2016242}.

\fi
% 
% \section{Beamlines}
% \label{sec:beamlines}
% \ifdefined\notFOREPJ
%  \input{content/beam_lines}
% \else
%  \input{beam_lines}
% \fi

\section{Experimental set-up}
\label{sec:tscope}
\ifdefined\notFOREPJ
% \input{content/tscope}
\else
 \input{tscope}
\fi

\section{Results and discussion}
\label{sec:discussion}
\ifdefined\notFOREPJ
% \input{content/discussion}
\else
 \input{discussion}
\fi


\section{Conclusion}
\label{sec:conclusion}
\ifdefined\notFOREPJ
 %
The ELAD sensor it's a new sensor concept that allows for the possibility to achieve an improved position resolution with no need of making a smaller pitch or increasing the number of readout channels. 
The resolution improvement in ELAD sensors is accomplished by increasing the lateral size of the charge distribution. 
For this purpose, deep implants are included in the ELAD sensors. 
The characteristic length, which affects the spatial resolution is the distance between the deep implants rather than the pitch size. 
Dedicated simulations, conducted in the TCAD framework, prove that the charge sharing in ELAD sensors is possible. 
A new plan for a production technique for multi-layered structures with implantation was discussed. 

\else
 
The ELAD sensor it's a new sensor concept that allows for the possibility to achieve an improved position resolution with no need of making a smaller pitch or increasing the number of readout channels. 
The resolution improvement in ELAD sensors is accomplished by increasing the lateral size of the charge distribution. 
For this purpose, deep implants are included in the ELAD sensors. 
The characteristic length, which affects the spatial resolution is the distance between the deep implants rather than the pitch size. 
Dedicated simulations, conducted in the TCAD framework, prove that the charge sharing in ELAD sensors is possible. 
A new plan for a production technique for multi-layered structures with implantation was discussed. 

\fi

%\ifdefined\notFOREPJ
%\else
\section*{Acknowledgements}
The test-beam support at DESY is highly appreciated. 
This work is supported by the Commission of the European Communities under the 6th Framework Programme 'structuring the European research area', contract number RII3-026126.
Furthermore, strong support from the Helmholtz Association and the BMBF is acknowledged.
%contract number RII3-CT-2006-026126.
%Funding agencies AIDA, EUDET

{\small
\section*{Data and materials}
The datasets supporting the conclusions of this article are available from reference \cite{jansen_data}.
The software used is available from the github repositories: 1) \url{https://github.com/eutelescope/eutelescope}, 2) \url{https://github.com/simonspa/eutelescope/}, branch \textit{scattering}
 and 3) \url{https://github.com/simonspa/resolution-simulator}.
%For the presented analysis, these specific tags have been used: \cite{jansen_2016_49065}.

% \section*{Competing interests}
% The authors declare that they have no competing interests.


\bibliographystyle{IEEEtran}
\ifdefined\notFOREPJ
\bibliography{bibtex/refs}
\else
\bibliography{refs}
\fi
}

\end{document}


\grid
