
The production of the ELAD sensor requires the developing of a new manufacturing technology. 
This new technology involves surface implantation and epitaxial growth. 
One of the methods of epitaxial growth~\cite{Lutz} is the Chemical Vapour Deposition (CVD) method. 
The CVD is the deposition of gaseous material onto a heated substrate. 
The deposition is achieved via the chemical reaction of the substructure with gaseous compounds. 
The surface implantation is realised by bombarding the substrate with accelerated ions.

The ELAD production uses the two techniques alternately. 
Initially, the first layer of deep implants is created by surface implantation on a p-type wafer. 
Then, on the implanted wafer creates an epitaxial layer, in the surface of this epitaxial zone, puts the second layer of deep implants and the process repeats. %FIXME bad english, rephrase

One possible risk of such a production is the ability of implants to diffuse during the CVD process. 
However, the TCAD SPROCESS simulation shows that the difference in sizes of deep implants before and after 20\,min heating up to $1100^\circ$C is less than $\SI{1}{\um}$. 
Therefore, %FIXME finish sentence
