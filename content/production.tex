
The ELAD-sensor production requires a new manufacturing technology to be developed. 
This new technology involves surface implantation and epitaxial growth. 
One of the methods of epitaxial growth~\cite{Lutz} is the Chemical Vapour Deposition (CVD) method. 
The CVD is the deposition of gaseous material onto a heated substrate. 
The deposition is achieved via the chemical reaction of the substructure with gaseous compounds. 
The surface implantation is realised by bombarding the substrate with accelerated ions.

The ELAD production uses the two techniques alternately. 
Initially, the first layer of deep implants is created by surface implantation on a p-type wafer. 
Than, an epitaxial layer is grown on the wafer. The second layer of the deep implants is created on top of the epitaxy one. This is accomplished by using the surface implantation. The process iterates until all the deep implants layers and the read-out implants are created. %FIXME bad english, rephrase

One possible risk of such a production is the ability of implants to diffuse during the CVD process. 
However, the TCAD SPROCESS simulation shows that the difference in sizes of deep implants before and after 20\,min heating up to $1100^\circ$C is less than $\SI{1}{\um}$. 
Therefore, it is necessary to consider this increase in the size during the sensor designing.%FIXME finish sentence
