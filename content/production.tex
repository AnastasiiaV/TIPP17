
The ELAD-sensor production requires a new manufacturing technology to be developed. 
The new technology involves surface implantation and epitaxial growing. 
The epitaxial growing consists in growing of the crystal in the correct lattice structure on top of a single-crystal wafer [3]. 
One of the methods of epitaxial growing is a Chemical Vapor Deposition (CVD) method. The CVD is the deposition of solid material onto a heated substrate. 
The deposition is conceived via the chemical reaction of the substructure with gaseous compounds, passing over it. 
The  CVD method requires the temperature of ~ $1100^\circ C$. The surface implantation is realized by bombarding the substrate with accelerated ions.

The ELAD production consists of several steps. Initially, the first layer of deep implants is generated by surface implantation on a p-type wafer. 
Then, on the implanted wafer creates an epitaxial layer, in the surface of this epitaxial zone, puts the second layer of deep implants and the process repeats. 

The critical side of such production is an ability of implants to diffuse during the CVD process. 
However, the TCAD SPROCESS simulation shows that the difference in sizes of deep implants before and after 20 min heating up to $1100^\circ C$ is less than 1 $\mu m$.
