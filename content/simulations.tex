For developing a new sensor is necessary to make a careful simulations. As a tool for simulations TCAD SYNOPSYS was selected. Technology Computer-Aided Design (TCAD) refers to the use of computer simulations to develop and optimize semiconductor processing technologies and devices [???]. There are three tools were used: SPROCESS, SDE and SDEVICE. SPROCESS simulates the fabrication steps in silicon process technologies in 2-D and 3-D. SDE is a 2-D/3-D device editor which builds and edits device structures using geometric operations. SDEVICE simulates the electrical, thermal, and optical characteristics of silicon and compound semiconductor devices in 2-D and 3-D [???].

The parameters which should be clarified are the width, depth of implants, distance within/to next layer, position/shift to neighboring layer, the number of layers and optimal doping concentrations for deep implants. As a simulation result, electric field profile for best charge sharing should be obtained. 

The TCAD geometry of the ELAD sensor is contain p-type sensor, 3 layers of n and p deep implants, p-spray isolation, readout implants, readout electrodes, $SiO_2$ and backplane. With the p-spray isolation technique a shallow unstructured p implant provides a doping density that, integrated over depth, exceeds somewhat the oxide surface charge density, thus preventing the buildup of an electron layer below the oxide [???]. The p-spray concentration was chosen according to the value of the breakdown voltage [???].