
Extensive simulation is essential for a new sensor development. 
The  Technology Computer-Aided Design (TCAD) SYNOPSYS was selected  as a tool for simulations, widely used for the development and optimization of semiconductor processing technologies and devices~\cite{Synopsys}. 
In particular, three tools have been exploited: SPROCESS, SDE and SDEVICE. 
%FIXME introduce variable for SPROCESS, SDE, SDEVICE, ...
SPROCESS simulates the fabrication steps in silicon process technologies. 
SDE builds and edits device structures using geometric operations. 
SDEVICE simulates the electrical, thermal, and optical characteristics of silicon and compound semiconductor devices~\cite{SynopsysIncG-2012.06}.

\begin{figure}[t]
\begin{minipage}[h]{0.39\linewidth}
\center{\includegraphics[trim= 0 0 0 0, height=7cm]{pictures/geometry.png} \\ (A)}
\end{minipage}
\begin{minipage}[h]{0.39\linewidth}
\center{\includegraphics[trim= 0 0 0 0, height=7cm]{pictures/elfield3.png} \\(B)}
\end{minipage}
\begin{minipage}[h]{0.2\linewidth}
\center{\includegraphics[trim= 0 0 0 0, width=2.5cm]{pictures/elfieldscale1.png}}
\end{minipage}
\caption[short description here]
 {(A) TCAD geometry of the ELAD sensor. 1 - p-spray, 2 - deep p-implant, 3 - deep n-implant, 4 - epi-zone. 
 (B) TCAD simulation of the electric field in the ELAD sensor. 
 }
\label{fig:geo-elfield}
\end{figure}

The parameters which should be optimised are the width and depth of implants, distance within/to the next layer, position/shift to neighbouring layer,
 the number of layers and optimal doping concentrations for deep implants.
The profile of the electric field yielding an optimal charge sharing is a result of the scan over the mentioned parameters. 

The TCAD geometry of the ELAD sensor Fig. \ref{fig:geo-elfield}~(A) contains the p-type sensor, three layers of n and p deep implants, the p-spray isolation, the readout implants, the readout electrodes, the SiO${}_2$ and backplane. 
With the p-spray isolation technique a shallow unstructured p implant prevents the build-up of an electron layer below the oxide~\cite{Lutz}. 
The p-spray concentration was chosen according to the value of the breakdown voltage~\cite{Pellegrini}. 
The shape of implants is described by using an error function. 
The total sensor thickness is 150 $\muup$m, the pitch size is 55 $\muup$m in order to match the TimePix3~\cite{TimePix}. readout chip, which is foreseen to be used for the test of the prototype sensors. 

The electric field simulation shows that the p- and n- deep implants lead to changes in the potential Fig.~\ref{fig:geo-elfield}~(B). %FIXME lead to cause ?!
The mean electric field is similar to that in a standard design sensor, but the deep implants change the field locally, i.e.\ induce an electric field component in $x$-direction. %FIXME induce, not include
The charges drift towards the middle between two strips/pixels and change their drift path due to diffusion. %FIXME ... the middle, %FIXME and change their drift path due to diffusion. not clear for the reader
The red zones in Fig.~\ref{fig:geo-elfield}~(B) indicate areas where the electric field forces electrons to change their drift path to the right, blue zones - to the left.
Consequently, the possibility of collecting the charge by two strips is given. 
%FIXME it is not the read zone that forces any one to do anything, rather: Red zones indicate areas where the eletric field blablabla and therfore electrons drift blablabla. 
In addition the simulation shows that the non-homogeneous electric field in $x$-direction is stable over time. 


\begin{figure}[t]
 \begin{minipage}[h]{0.24\linewidth}
 \center{\includegraphics[trim= 0 0 0 0, height=7cm]{pictures/driftt1_1.png} \\ (1)}
 \end{minipage}
 \begin{minipage}[h]{0.24\linewidth}
 \center{\includegraphics[trim= 0 0 0 0, height=7cm]{pictures/driftt2_1.png} \\(2)}
 \end{minipage}
 \begin{minipage}[h]{0.24\linewidth}
 \center{\includegraphics[trim= 0 0 0 0, height=7cm]{pictures/driftt3_1.png} \\(3)}
 \end{minipage}
 \begin{minipage}[h]{0.24\linewidth}
 \center{\includegraphics[trim= 0 0 0 0, width=2cm]{pictures/driftscale1.png}}
 \end{minipage}
\caption[short description here]
  {TCAD drift simulation in the ELAD sensor. (1) at $t = \SI{e-12}{\sec}$, (2) at $t = \SI{e-10}{\s}$, (3) at $t = \SI{1.2e-9}{\s}$.}
\label{fig:drift}
\end{figure} %FIXME pictures are of bas quality, use high resolution! or at best vector grpahics (eps file)

To understand the behaviour of charge carriers in the ELAD sensor a drift simulation has been carried out. 
It shows that the charge carriers created near an electrode are collected by this electrode, but the charge created beneath the deep implants area %FIXME It shows (GRAMMAR: because the simluations still show that, they have not stopped showing that)
 changes its drift path and is collected by two strips Fig.~\ref{fig:drift}.  
Thereby, the drift simulation supports the basic concept of charge sharing in ELAD sensors making use of a non-homogeneous electric field in $x$-direction. 


 